%%%%%%%%%%%%%%%%%%%%%%%%%%%%%%%%%%%%%%%%%
% Medium Length Professional CV
% LaTeX Template
% Version 2.0 (8/5/13)
%
% This template has been downloaded from:
% http://www.LaTeXTemplates.com
%
% Original author:
% Trey Hunner (http://www.treyhunner.com/)
%
% Important note:
% This template requires the resume.cls file to be in the same directory as thehttps://preview.overleaf.com/public/czwvvxzknjqj/images/fa6f8726841f80932e8ac6e4de79722de090a3cc.jpeg
% .tex file. The resume.cls file provides the resume style used for structuring the
% document.
%
%%%%%%%%%%%%%%%%%%%%%%%%%%%%%%%%%%%%%%%%%

%----------------------------------------------------------------------------------------
%	PACKAGES AND OTHER DOCUMENT CONFIGURATIONS
%----------------------------------------------------------------------------------------

\documentclass{resume} % Use the custom resume.cls style

\usepackage[left=0.7in,top=0.7in,right=0.7in,bottom=0.7in]{geometry} % Document margins
\usepackage{hyperref}
\newcommand{\tab}[1]{\hspace{.2667\textwidth}\rlap{#1}}
\newcommand{\itab}[1]{\hspace{0em}\rlap{#1}}
\name{ZENG, YUKUN} % Your name
\address{+1 (979)739 9315 \\ yzeng@tamu.edu}
\address{1600 Southwest Pkwy APT 1301 \\ College Station, TX 77840, U.S.} % Your address
%\address{123 Pleasant Lane \\ City, State 12345} % Your secondary addess (optional)
 % Your phone number and email

\begin{document}
\begin{rSection}{Objective}
Seeking full-time software/research internship for 2017 summer, see \href{https://github.com/vincenttsang}{Github} and \href{https://parasol.tamu.edu/~yzeng/}{Homepage} for more about me
\end{rSection}
%----------------------------------------------------------------------------------------
%	EDUCATION SECTION
%----------------------------------------------------------------------------------------

\begin{rSection}{Education}
{\bf Texas A\&M University} \hfill {\bf College Station, TX} 
\\ Master of Science in Computer Science \hfill {\em Expected Graduation: May, 2018}
\\ GPA: 3.67/4.00
\vspace*{-0.2em}

{\bf Harbin Institute of Technology} \hfill {\bf Weihai, China} 
\\ Bachelor of Engineering in Software Engineering \hfill {\em Sep. 2012--July 2016}
\\ GPA: Overall 3.45/4.00, Major 3.70/4.00

%Minor in Linguistics \smallskip \\
%Member of Eta Kappa Nu \\
%Member of Upsilon Pi Epsilon \\


\end{rSection}


%----------------------------------------------------------------------------------------
%	TECHNICAL STRENGTHS SECTION
%----------------------------------------------------------------------------------------

%\begin{rSection}{Campus Services}
%{\bf President} \hfill {HIT Robot Innovation Workshop} 
%\\{\bf Teaching Assistant} \hfill {School of Computer Sci. and Tech., HIT} 
%\\{\bf Research Assistant} \hfill {HIT-NS Joint R\&D Center} 
%\\{\bf Student Engineer} \hfill {HIT Network Center} 
% Technical Strenths
%\begin{tabular}{ @{} >{\bfseries}l @{\hspace{6ex}} l }
%Computer Languages &  C/C++, MATLAB \\
%Software \& Tools & HTML, LaTeX, Excel, Gerris, Mathematica, ASPEN Plus, Tecplot \\
%\end{tabular}

%\end{rSection}

%----------------------------------------------------------------------------------------
%	WORK EXPERIENCE SECTION
%----------------------------------------------------------------------------------------
\begin{rSection}{Research Experience}
\begin{rSubsection}{Graduate Researcher}{College Station, TX}{Parasol Lab, supervised by Prof. Nancy M. Amato}{Sep. 2016 - Present}
\item Implemented simple EST motion planner and improved it with single-shot midpoint guided sampling
\item Experiences in working with robotic fundamentals libraries, like \href{https://parasol.tamu.edu/groups/amatogroup/research/UserGuided/Old/vizmo++/}{VIZMO++}, PMPL(Parasol Motion Planning Library), and with parallel computing library (\href{https://parasol.tamu.edu/groups/rwergergroup/research/stapl/}{STAPL}) for improving motion planning performance
\item Generalizing embedding graph, flow graph and dynamic region utilities used in \href{http://wafr2016.berkeley.edu/papers/WAFR_2016_paper_36.pdf}{Dynamic Region-biased RRT}
\end{rSubsection}

\begin{rSubsection}{Team Leader}{Weihai, China}{HIT Robot Innovation Lab}{May 2015 - Jan. 2016}
\item Led the development of two competition projects in National Robot Championship
\item Used RoboBasic to develop a matrix approach of stable robot gait planning for RoboNova series robots
\end{rSubsection}

\end{rSection}

\begin{rSection}{Work Experience}
\begin{rSubsection}{Software Engineer Intern}{Shenzhen, China}{ARRIS Group}{Jan. - May 2016}
\item Automated signal-free wireless testing environment setup and modem performance benchmarking
\item Modem routing architecture modification for security reinforcement
\end{rSubsection}

\begin{rSubsection}{Technical Solution Intern}{Dalian, China}{Neusoft}{July 2014 - Aug. 2014}
\item Neusoft IM (Instant Message) System development in Java with multithread chating, socket communication, real-time server monitoring and persistent data storage using Oracle DB
\end{rSubsection}
\end{rSection}



\begin{rSection}{Selected Projects}
\begin{rSubsection}{\href{http://ieeexplore.ieee.org/document/7335296/?arnumber=7335296}{Mobile Storm: Distributed Real-time Stream Processing for Mobile Cloud}}{Research Assistant}{}{}
\item Proposed greedy and genetic algorithms for job topology allocation on multi workers with varying executors (NP-Hard), generally yields results within $30\%$ gap comparing to near-optimal solution (CPLEX) but runs 100x faster
\item Developed a Neural-like topology generator for job submission simulation to test our allocation algorithm
\item Integrating facial processing utilities (including face detection, recognition, tracking, etc) and distributed computational tasks on MobiStorm platform, involving socket communication, multi-threading, stream processing, etc

\end{rSubsection}

\begin{rSubsection}{\href{http://ieeexplore.ieee.org/document/7579961/}{Hi-Responsive Scheduling with MR Performance Prediction on Hadoop YARN}}{Research Assistant}{}{}
\item Experiences in Hadoop YARN basics like cluster setup and maintenance, Hadoop APIs (including RESTful APIs)
\item Developed a cloud computing benchmark suite to test performance of heterogeneous Hadoop YARN cluster running multi frameworks like MapReduce, Spark, etc
\item Proposed a novel job size prediction approach based on Machine Learning techniques and designed a size-based scheduling framework, which substantially improved the responsiveness of Hadoop cluster

\end{rSubsection}


%------------------------------------------------

%\begin{rSubsection}{Motion Planning\&RoboDance Implementation}{May 2015 - Jan 2016}{Leader}{HIT Robot Innovation Workshop}
%\item Implemented a robot motion planning system and ensured the robot stability in the pose switch process.
%\item Programmed humanoid robot dance and solved the synchronization problem among multiple robots.
%\end{rSubsection}

%------------------------------------------------

\begin{rSubsection}{Flash Vocabulary - Lightweight website for boosting vocabulary online}{Leader}{}{}
\item Devised a novel MVC-derived pattern that best fits the interaction mode of our website
\item Developed a comprehensive front-end framework to simplify front-end codes and create a universal UI
\item Adopted AJAX and HTML5 Local Storage to avoid unnecessary reloading and enhance user experience
\end{rSubsection}

%------------------------------------------------

\begin{rSubsection}{Jizhi Tutor Service - Online edu platform on Cloud}{Co-Leader}{}{}
\item Applied HTML, CSS, Javascript (JQuery) to the front-end dev, used complex SQL Server database (with triggers, view, stored procedure, etc) and .NET platform for data storage and business logic
\item Lead the entire platform dev from designing, implementation to Cloud deployment and maintenance
\end{rSubsection}

%------------------------------------------------

\begin{rSubsection}{General Coding - An APP to improve programmers' productivity}{Key Developer}{}{}
\item Developed the APP which highlights on improving user experience by optimizing data structure and algorithms, extensible to multi programming language API integration
\item Implemented an objective linked-list and fuzzy query algorithm (Levenshtein Distance) in our Full-Text
Inter-PL (Programming Language) API search engine
\end{rSubsection}


\end{rSection}

%----------------------------------------------------------------------------------------
%\begin{rSection}{Activities}
%{\bf Grader} \hfill {CSE, TAMU}\\
%{\bf Member} \hfill {Parasol Lab, TAMU}\\
%{\bf President} \hfill {Robot Innovation Workshop, HIT}\\
%{\bf Volunteer} \hfill {MSRA Language Tech Summer School}\\
%{\bf Student Engineer} \hfill {Network Center, HIT}



% Relevant Courses
%\itab{\textbf{Core Courses}} \tab{}  \tab{\textbf{Other Courses}}
%\\ \itab{Fluid Mechanics \& its applications } \tab{}  \tab{Computational Methods in Engineering}
%\\ \itab{Thermodynamics} \tab{}  \tab{Fundamental of Computing} 
%\\ \itab{Heat Transfer \& its applications} \tab{}  \tab{Probability and Statistics} 
%\\ \itab{Mass Transfer \& its applications} \tab{} \tab{Calculus \& Linear Algebra}
%\\ \itab{Transport Phenomena (ongoing)} \tab{} \tab{Introduction to Mechanics}
% \\ \itab{Process Control (ongoing)} \tab{} \tab{Electrodynamics}

%\end{rSection}

\begin{rSection}{Course Projects}
{\bf CSCE 614 Computer Architecture}\hfill {\emph Daniel Jimenez}\\
Cache behavior simulator with LRU and random replacement\\
\href{https://github.com/vincenttsang/CompArch/blob/master/branch-predictor/p1-final-report/p1-final-report.pdf}{Fast path-based neural branch predictor with perceptron}\\
\href{https://github.com/vincenttsang/CompArch/blob/master/cache-replacer/p2-final-report/p2-final-report.pdf}{High performance cache replacement using re-reference interval prediction (RRIP)}
\end{rSection}

\begin{rSection}{Publications}

{[1]Liu, Yang, }\underline{\bf Yukun Zeng}{ and Xuefeng Piao. ``High-Responsive Scheduling with MapReduce Performance Prediction on Hadoop YARN.'' Embedded and Real-Time Computing Systems and Applications (RTCSA), 2016 IEEE 22nd International Conference on. IEEE, 2016.
}
\vspace*{-0.2em}

{[2]Gaoyang Li, Guangri Quan, }\underline{\bf Yukun Zeng}{, ``MASS: A short reads alignment tool oriented to massive}{data,'' The Workshop on Algorithms in Bioinformatics 2016, submitted.}

\end{rSection}

\begin{rSection}{Honors\&Awards}
{\bf Best Paper Award}{ for Outstanding Bachelor Dissertation} \hfill {July 2016}
\vspace*{-0.4em}

{\bf Meritorious Winner(1st Prize)}{ in National Robot Championship} \hfill {July 2015}
\vspace*{-0.4em}

{\bf Honorable Mention}{ in Mathematical Contest in Modeling (MCM)} \hfill {Apr. 2015}
\vspace*{-0.4em}

{\bf 2nd Place}{ in HIT Software Design Competition} \hfill {Mar. 2014} 
\vspace*{-0.4em}

{\bf People's Scholarship} 5 times \hfill {Oct. 2013, Oct. 2014, May 2015, Oct. 2015, Apr. 2016}
\vspace*{-0.4em}

\end{rSection}

\begin{rSection}{Professional Activities}
{\bf Student Volunteer}\hfill Weihai, China\\
HIT-MSRA Language Technology Summer School\hfill {\emph July, 2013}
\vspace*{-0.2em}

{\bf Peer Reviewer}\hfill College Station, TX\\
IEEE Transactions on Robotics (T-RO)\\
IEEE Robotics and Automation Letters (RA-L)\\
Springer Journal of Intelligent \& Robotic Systems (JINT)\\
ACM Transactions on Spatial Algorithms and Systems (TSAS)\\
IEEE International Conference on Robotics and Automation (ICRA)\hfill {\emph Sep. 2016 -- Present}
\end{rSection}

\begin{rSection}{Teaching Experience}
\begin{rSubsection}{Grader}{College Station, TX}{Texas A\&M University}{Sep. - Dec. 2016}
\item Test case development and test automation for \href{http://faculty.cse.tamu.edu/dilma/web-csce410-fall16/index.htm}{CSCE410 Operating Systems}, involving frame management, memory paging, virtual memory, thread scheduling, device driver and file systems
\end{rSubsection}

\begin{rSubsection}{Teaching Assistant}{Weihai, China}{Harbin Inst. of Tech.}{Sep. 2014 - June 2016}
\item Including the following major courses: Database Systems, Computer Networking, Operating Systems, Compiler Principles, Object-Oriented Programming, etc
\end{rSubsection}
\end{rSection}

\end{document}
