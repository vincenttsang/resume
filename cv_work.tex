%%%%%%%%%%%%%%%%%%%%%%%%%%%%%%%%%%%%%%%%%
% Medium Length Professional CV
% LaTeX Template
% Version 2.0 (8/5/13)
%
% This template has been downloaded from:
% http://www.LaTeXTemplates.com
%
% Original author:
% Trey Hunner (http://www.treyhunner.com/)
%
% Important note:
% This template requires the resume.cls file to be in the same directory as thehttps://preview.overleaf.com/public/czwvvxzknjqj/images/fa6f8726841f80932e8ac6e4de79722de090a3cc.jpeg
% .tex file. The resume.cls file provides the resume style used for structuring the
% document.
%
%%%%%%%%%%%%%%%%%%%%%%%%%%%%%%%%%%%%%%%%%

%----------------------------------------------------------------------------------------
%	PACKAGES AND OTHER DOCUMENT CONFIGURATIONS
%----------------------------------------------------------------------------------------

\documentclass{resume} % Use the custom resume.cls style

\usepackage[left=0.7in,top=0.7in,right=0.7in,bottom=0.7in]{geometry} % Document margins
\newcommand{\tab}[1]{\hspace{.2667\textwidth}\rlap{#1}}
\newcommand{\itab}[1]{\hspace{0em}\rlap{#1}}
\name{ZENG, YUKUN} % Your name
\address{+1 (979)739 9315 \\ yzeng@tamu.edu}
\address{1600 Southwest Pkwy APT 1301 \\ College Station, TX 77840, U.S.} % Your address
%\address{123 Pleasant Lane \\ City, State 12345} % Your secondary addess (optional)
 % Your phone number and email

\begin{document}

%----------------------------------------------------------------------------------------
%	EDUCATION SECTION
%----------------------------------------------------------------------------------------

\begin{rSection}{Education}
{\bf Texas A\&M University} \hfill {\bf College Station, TX} 
\\ Master of Science in Computer Science \hfill {\em Expected Graduation: July, 2018}
\vspace*{-0.2em}

{\bf Harbin Institute of Technology} \hfill {\bf Weihai, China} 
\\ Bachelor of Engineering in Software Engineering \hfill {\em Sep. 2012--July 2016}
\\ GPA: Overall 3.45/4.00, Major 3.70/4.00

%Minor in Linguistics \smallskip \\
%Member of Eta Kappa Nu \\
%Member of Upsilon Pi Epsilon \\


\end{rSection}
%----------------------------------------------------------------------------------------
%	TECHNICAL STRENGTHS SECTION
%----------------------------------------------------------------------------------------

%\begin{rSection}{Campus Services}
%{\bf President} \hfill {HIT Robot Innovation Workshop} 
%\\{\bf Teaching Assistant} \hfill {School of Computer Sci. and Tech., HIT} 
%\\{\bf Research Assistant} \hfill {HIT-NS Joint R\&D Center} 
%\\{\bf Student Engineer} \hfill {HIT Network Center} 
% Technical Strenths
%\begin{tabular}{ @{} >{\bfseries}l @{\hspace{6ex}} l }
%Computer Languages &  C/C++, MATLAB \\
%Software \& Tools & HTML, LaTeX, Excel, Gerris, Mathematica, ASPEN Plus, Tecplot \\
%\end{tabular}

%\end{rSection}

%----------------------------------------------------------------------------------------
%	WORK EXPERIENCE SECTION
%----------------------------------------------------------------------------------------
\begin{rSection}{Experience}
\begin{rSubsection}{Graduate Researcher}{College Station, TX}{Parasol Lab, supervised by Prof. Nancy M. Amato}{Sep. 2016 - Present}
\item Developed a self-motivated Robotics Motion Planner with Midpoint Guided Sampling
\item Researching on simulating robot group behavior, e.g., Multi Robot Persistent Coverage Problem simulation
\end{rSubsection}
\begin{rSubsection}{Grader}{College Station, TX}{CSCE 410 Operating Systems, Prof. Dilma Da Silva}{Sep. 2016 - Present}
\item Developed test case for grading projects, built automated testing environment, answered students' questions
\end{rSubsection}

\begin{rSubsection}{Software Engineer Intern}{Shenzhen, China}{ARRIS Group}{Jan. - May 2016}
\item Benchmarked network performance of our product to track performance bottlenecks and improve modem bandwidth
\item Revised routing architecture to achieve proper LAN connectivity and meantime ensure overall security
\end{rSubsection}
\begin{rSubsection}{Co-Chair}{Weihai, China}{HIT Robot Innovation Workshop}{May 2015 - Jan. 2016}
\item Led the development of two competition projects in National Robot Championship
\item Used RoboBasic to develop a matrix approach of stable robot gait planning for RoboNova series robots
\end{rSubsection}
\end{rSection}

\begin{rSection}{Selected Projects}
\begin{rSubsection}{Mobile Storm: Distributed real-time stream processing for mobile cloud}{Research Assistant}{}{}
\item Proposed greedy and genetic algorithms for job topology allocation on multi workers with varying executors (NP-Hard), generally yields results within $30\%$ gap comparing to near-optimal solution (CPLEX) but runs 100x faster
\item Developed a Neural-like topology generator for job submission simulation to test our allocation algorithm
\item Working on distributed streaming face recognition applications for real-world application experiments

\end{rSubsection}

\begin{rSubsection}{High-responsive scheduling for heterogeneous Hadoop YARN cluster}{Research Assistant}{}{}
\item Developed a cloud computing benchmark suite to test performance of heterogeneous Hadoop YARN cluster running multi frameworks like MapReduce, Spark, etc
\item Proposed a novel job size prediction approach based on Machine Learning techniques and designed a size-based scheduling framework, which substantially improved the responsiveness of Hadoop cluster

\end{rSubsection}


%------------------------------------------------

%\begin{rSubsection}{Motion Planning\&RoboDance Implementation}{May 2015 - Jan 2016}{Leader}{HIT Robot Innovation Workshop}
%\item Implemented a robot motion planning system and ensured the robot stability in the pose switch process.
%\item Programmed humanoid robot dance and solved the synchronization problem among multiple robots.
%\end{rSubsection}

%------------------------------------------------

\begin{rSubsection}{Flash Vocabulary - Lightweight website for boosting vocabulary online}{Leader}{}{}
\item Devised a novel MVC-derived pattern that best fits the interaction mode of our website
\item Developed a comprehensive front-end framework to simplify front-end codes and create a universal UI
\item Adopted AJAX and HTML5 Local Storage to avoid unnecessary reloading and enhance user experience
\end{rSubsection}

%------------------------------------------------

\begin{rSubsection}{Jizhi Tutor Service - Online edu platform on Cloud}{Co-Leader}{}{}
\item Applied HTML, CSS, Javascript (JQuery) to the front-end dev, used complex SQL Server database (with triggers, view, stored procedure, etc) and .NET platform for data storage and business logic
\item Lead the entire platform dev from designing, implementation to Cloud deployment and maintenance
\end{rSubsection}

%------------------------------------------------

\begin{rSubsection}{General Coding - An APP to improve programmers' productivity}{Key Developer}{}{}
\item Developed the APP which highlights on improving user experience by optimizing data structure and algorithms, extensible to multi programming language API integration
\item Implemented an objective linked-list and fuzzy query algorithm (Levenshtein Distance) in our Full-Text
Inter-PL (Programming Language) API search engine
\end{rSubsection}


\end{rSection}

%----------------------------------------------------------------------------------------
%\begin{rSection}{Activities}
%{\bf Grader} \hfill {CSE, TAMU}\\
%{\bf Member} \hfill {Parasol Lab, TAMU}\\
%{\bf President} \hfill {Robot Innovation Workshop, HIT}\\
%{\bf Volunteer} \hfill {MSRA Language Tech Summer School}\\
%{\bf Student Engineer} \hfill {Network Center, HIT}



% Relevant Courses
%\itab{\textbf{Core Courses}} \tab{}  \tab{\textbf{Other Courses}}
%\\ \itab{Fluid Mechanics \& its applications } \tab{}  \tab{Computational Methods in Engineering}
%\\ \itab{Thermodynamics} \tab{}  \tab{Fundamental of Computing} 
%\\ \itab{Heat Transfer \& its applications} \tab{}  \tab{Probability and Statistics} 
%\\ \itab{Mass Transfer \& its applications} \tab{} \tab{Calculus \& Linear Algebra}
%\\ \itab{Transport Phenomena (ongoing)} \tab{} \tab{Introduction to Mechanics}
% \\ \itab{Process Control (ongoing)} \tab{} \tab{Electrodynamics}

%\end{rSection}
\begin{rSection}{Publications}

{[1]Liu, Yang, }{\bf Yukun Zeng}{ and Xuefeng Piao. ``High-Responsive Scheduling with MapReduce Performance Prediction on Hadoop YARN.'' Embedded and Real-Time Computing Systems and Applications (RTCSA), 2016 IEEE 22nd International Conference on. IEEE, 2016.
}
\vspace*{-0.2em}

{[2]Gaoyang Li, Guangri Quan, }{\bf Yukun Zeng}{, ``MASS: A short reads alignment tool oriented to massive}{data,'' The Workshop on Algorithms in Bioinformatics 2016, submitted.}

\end{rSection}


\begin{rSection}{Honors\&Awards}

{\bf 1st Prize in National Robot Championship} \hfill {July 2015}
\vspace*{-0.4em}

{\bf Honorable Mention in Mathematical Contest in Modeling (MCM)} \hfill {Apr. 2015}
\vspace*{-0.4em}

{\bf 2nd Prize in Software Design Competition} \hfill {Mar. 2014} 
\vspace*{-0.4em}

{\bf People's Scholarship} 5 times \hfill {Oct. 2013, Oct. 2014, May 2015, Oct. 2015, Apr. 2016}
\vspace*{-0.4em}

\end{rSection}

\end{document}
